% Documentation for 1-phase flow simulator.
% Copyright (C) 2005 Alexey Dobriyan <adobriyan@gmail.com>
%
% This information is free; you can redistribute it and/or modify
% it under the terms of version 2 of the GNU General Public License as
% published by the Free Software Foundation.
%
% This work is distributed in the hope that it will be useful,
% but WITHOUT ANY WARRANTY; without even the implied warranty of
% MERCHANTABILITY or FITNESS FOR A PARTICULAR PURPOSE.  See the
% GNU General Public License for more details.
%
% You should have received a copy of the GNU General Public License
% along with this work; if not, write to the Free Software
% Foundation, Inc., 51 Franklin St, Fifth Floor, Boston, MA  02110-1301  USA

\documentclass{article}
\begin{document}
\title{1-phase flow}
\author{Alexey Dobriyan $<$adobriyan@gmail.com$>$}
\maketitle

\section{Differential equations}

Continuity equation:
\begin{equation}
\frac{\partial (\phi \rho)}{\partial t} + \nabla (\rho \vec{v}) = q
\end{equation}
where $\phi = \phi(\vec{r}, p)$ --- porosity of the medium, $\rho = \rho(p)$
--- density of the fluid, $t$~--- time, $\vec{v} = \vec{v}(\vec{r}, t)$ ---
flow velocity, $q = q(\vec{r}, t)$ --- fluid injection rate.

Darcy's law:
\begin{equation}
\vec{v} = -\frac{K}{\mu} (\nabla p - \rho g \nabla D)
\end{equation}
$K = K(\vec{r}, p)$ --- permeability, $\mu = \mu(p)$ --- viscosity, $p =
p(\vec{r}, t)$ --- pressure in the fluid, $g$ --- acceleration due to gravity,
$D = D(\vec{r})$ --- depth function.
\begin{equation}
\frac{\partial (\phi \rho)}{\partial t} = \nabla \left(\frac{\rho K}{\mu}
(\nabla p - \rho g \nabla D) \right) + q
\end{equation}

\section{Assumptions}
\begin{enumerate}
\item $D$ = const
\begin{equation}
\frac{\partial (\phi \rho)}{\partial t} = \nabla \left(\frac{\rho K}{\mu}
\nabla p \right) + q
\end{equation}
\item $\phi$ = const
\begin{equation}
\phi \frac{\partial \rho}{\partial t} = \nabla \left(\frac{\rho K}{\mu}
\nabla p \right) + q
\end{equation}
\item $K$ = const
\begin{equation}
\phi \frac{\partial \rho}{\partial t} = K \nabla \left(\frac{\rho}{\mu} \nabla
p \right) + q
\end{equation}
\item $\mu$ = const
\begin{equation}
\phi \frac{\partial \rho}{\partial t} = \frac{K}{\mu} \nabla (\rho \nabla p) +
q
\end{equation}

\begin{equation}
\frac{\partial \rho}{\partial t} = \rho' \frac{\partial p}{\partial t}
\end{equation}
\begin{equation}
\nabla (\rho \nabla p) = \nabla \rho \nabla p + \rho \Delta p = \rho' (\nabla
p)^2 + \rho \Delta p
\end{equation}
\begin{equation}
\phi \rho' \frac{\partial p}{\partial t} = \frac{K}{\mu} \left(\rho' (\nabla
p)^2 + \rho \Delta p \right) + q
\end{equation}
\begin{equation}
\phi \frac{\rho'}{\rho} \frac{\partial p}{\partial t} = \frac{K}{\mu}
\left(\frac{\rho'}{\rho} (\nabla p)^2 + \Delta p \right) + \frac{q}{\rho}
\end{equation}
\item $\frac{\rho'}{\rho}$ = $c$ = const
\begin{equation}
\phi c \frac{\partial p}{\partial t} = \frac{K}{\mu} \left(c (\nabla p)^2 +
\Delta p \right) + \frac{q}{\rho}
\end{equation}
\item $c (\nabla p)^2 \ll \Delta p$ --- fluid of slight compressibility
\begin{equation}
\phi c \frac{\partial p}{\partial t} = \frac{K}{\mu} \Delta p + \frac{q}{\rho}
\end{equation}
\item $q$ = 0
\begin{equation}
\frac{\partial p}{\partial t} = \frac{K}{\phi \mu c} \Delta p
\end{equation}
\item 1-dimensional space
\begin{equation}
\frac{\partial p}{\partial t} = \frac{K}{\phi \mu c} \frac{\partial^2
p}{\partial x^2}
\end{equation}
where $p = p(x, t)$.
\end{enumerate}

\section{Initial and boundary conditions}
\begin{equation}
p(x, 0) = p_{x 0}(x), 0 \leq x \leq L
\end{equation}
\begin{equation}
p(0, t) = p_{l t}(t), t \geq 0
\end{equation}
\begin{equation}
p(L, t) = p_{r t}(t), t \geq 0
\end{equation}

\section{Finite Difference Scheme}

Let's use the simplest explicit finite difference scheme on uniform grid for
now. $\Delta x$, $\Delta t$ are constants. Second deriative in space is
evaluated from values of 3 neighbor points.
\begin{equation}
p_{i j} = p(i \Delta x, j \Delta t), 0 \leq i \leq N - 1, j \geq 0
\end{equation}
\begin{equation}
p_{i 0} = p_{x 0}(i \Delta x), 0 \leq i \leq N - 1
\end{equation}
\begin{equation}
p_{0 j} = p_{l t}(j \Delta t), j \geq 0
\end{equation}
\begin{equation}
p_{N - 1, j} = p_{r t}(j \Delta t), j \geq 0
\end{equation}
\begin{equation}
\frac{p_{i, j + 1} - p_{i j}}{\Delta t} = \frac{K}{\phi \mu c} \frac{p_{i - 1,
j} - 2 p_{i j} + p_{i + 1, j}}{(\Delta x)^2}, 1 \leq i \leq N - 2, j \geq 1
\end{equation}

\section{Numerical Stability}

\begin{equation}
\frac{\partial p}{\partial t} = a^2 \frac{\partial^2 p}{\partial x^2}
\end{equation}
\begin{equation}
\frac{p_{m, n + 1} - p_{m n}}{\Delta t} = a^2 \frac{p_{m - 1, n} - 2 p_{m n} +
p_{m + 1, n}}{(\Delta x)^2}
\end{equation}
\begin{equation}
p_{m n} = \gamma^n e^{i p m \Delta x}
\end{equation}
\begin{equation}
\frac{\gamma^{n + 1} e^{i p m \Delta x} - \gamma^n e^{i p m \Delta x}}{\Delta
t} = a^2 \frac{\gamma^n e^{i p (m - 1) \Delta x} - 2 \gamma^n e^{i p m \Delta
x} + \gamma^n e^{i p (m + 1) \Delta x}}{(\Delta x)^2}
\end{equation}
$\frac{\gamma - 1}{\Delta t} = a^2 \frac{e^{-i p \Delta x} - 2 + e^{i p \Delta
x}}{(\Delta x)^2} = 2 a^2 \frac{cos(p \Delta x) - 1}{(\Delta x)^2} = -4 a^2
\frac{\sin^2 \frac{p \Delta x}{2}}{(\Delta x)^2}$
\begin{equation}
\gamma = 1 - 4 a^2 \frac{\Delta t}{(\Delta x)^2} \sin^2 \frac{p \Delta x}{2}
\end{equation}

For a finite difference scheme to be numerically stable, $|\gamma|$ must be
less or equal to 1.
\begin{equation}
-1 \leq 1 - 4 a^2 \frac{\Delta t}{(\Delta x)^2} \sin^2 \frac{p \Delta x}{2}
\leq 1
\end{equation}
\begin{equation}
\frac{\Delta t}{(\Delta x)^2} \leq \frac{1}{2 a^2 \sin^2 \frac{p \Delta x}{2}}
\end{equation}
or even
\begin{equation}
\frac{\Delta t}{(\Delta x)^2} \leq \frac{1}{2 a^2}
\end{equation}

In the simplest case:
\begin{equation}
a^2 = \frac{K}{\phi \mu c}
\end{equation}

\section{COPYING}

\hyphenation{MER-CHAN-TA-BI-LITY}

This information is free; you can redistribute it and/or modify
it under the terms of version 2 of the GNU General Public License as
published by the Free Software Foundation.

This work is distributed in the hope that it will be useful,
but WITHOUT ANY WARRANTY; without even the implied warranty of
MERCHANTABILITY or FITNESS FOR A PARTICULAR PURPOSE.  See the
GNU General Public License for more details.

You should have received a copy of the GNU General Public License
along with this program; if not, write to the Free Software
Foundation, Inc., 51 Franklin St, Fifth Floor, Boston, MA  02110--1301  USA

\end{document}
